\chapter*{Summary}\markboth{Summary}{Summary}
\addcontentsline{toc}{chapter}{Summary}

Events are getting more popular and more frequent in cities around the world. In the Netherlands in 2017, the number of festivals grew to almost 1000 \footnote{https://www.eventbranche.nl/nieuws/aantal-festivals-groeit-tot-bijna-1000-per-jaar-aantal-bezoeken-daalt-miniem-16483.html}. These events take place in large areas of the city, they have a common topic, they include sub-events (activities), and they have start and end times and lasts from one day to several days. Examples of events are the national holidays, Soul Live Festival and trade exhibitions. City events can easily attract a large number of people. Event stakeholders, such as the event organizers, police, municipalities and other authorities, and crowd managers are concerned with guaranteeing the safety, comfort and general well being of the attendees. To this end, they enforce predefined crowd management measures that are adaptive to the current state of the event environment and of the participating crowd. This state is measured through information about the factors influencing event planning \citep{li2019crowds} and pedestrian behaviour \citep{still2000crowd,tubbs2007egress,abbott2000event, zomer2015managing} for crowd management, such as crowd size, density, mobility, emotion, visitor profile, and location. Conventionally, this information is derived from data provided by stewards (operating on the ground during the event) and sometimes pre-installed sensing infrastructures, such as counting systems, Bluetooth/ Wi-Fi sensors, and video cameras. While effective, these solutions suffer from several issues: they provide little information about sentiments, gender and age distribution, they are expensive, they cannot provide Spatio-temporal information, and they are complex to install and maintain.  

