\chapter*{Glossary}\markboth{Glossary}{Glossary}
\addcontentsline{toc}{chapter}{Glossary}

%\renewcommand{\imagepath}[1]{\fname{0-misc/images/#1}}
\graphicspath{{./0-misc/images/}}


\section*{Conventions}
The following conventions are used in this thesis for notation
and symbols:
\begin{itemize}
\item A lower case character typeset in boldface, e.g., $\mathbf{x}$,
   represents a column vector.
\item The number of elements in a vector $\mathbf{x}$ is indicated by
   $n_{\mathbf{x}}$.
\item \ldots
\end{itemize}

\section*{List of symbols and notations}
Below follows a list of the most frequently used symbols and notations
in this thesis. Symbols particular to power network applications
are explained only in the relevant chapters.

\vspace*{-0.4cm}\begin{tabbing}
\hspace*{3.5cm}\=\kill\\
$ \mathbf{A}  $ \> system matrices of linear time-invariant models \\
\\[-0.1cm]
$ \mathbf{B}  $ \> input matrices of linear time-invariant models \\
\\[-0.1cm]
$ \mathbf{E}_1 $ \> matrices of mixed-logical dynamic models \\
\end{tabbing}

\vspace*{-0.7cm}
\section*{List of abbreviations}
The following abbreviations are used in this thesis:
\begin{tabbing}
\hspace*{4cm}\=\kill
AVR \> Automatic Voltage Regulator\\
DAE \> Differential-Algebraic Equations\\
FACTS \> Flexible Alternating-Current Transmission System\\
MPC \> Model Predictive Control\\
\ldots \> \ldots
\end{tabbing}
