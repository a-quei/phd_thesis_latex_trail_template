%%%%%%%%%%%%%%%%%%%%%%%%%%%%%%%%%%%%%%%%%%%%%%%%%%%%%%%%%%%%%%%%%%%%%%%

% make sure that "1-intro/images" refers to the subdirectory in which
% you store images for this chapter, relative to the base directory of
% your thesis (which contains the main thesis.tex file)

%\renewcommand{\imagepath}[1]{\fname{1-intro/images/#1}}
%\graphicspath{{./1-intro/images/}}

%%%%%%%%%%%%%%%%%%%%%%%%%%%%%%%%%%%%%%%%%%%%%%%%%%%%%%%%%%%%%%%%%%%%%%%

\chapter{Introduction}\label{ch1_title}

%In this chapter we present the background and the motivation for the
%research addressed in this thesis.  In Section \ref{transintro} we
%\ldots In Section \ref{overviewandroadmap} we conclude the chapter
%with an overview and road map of this thesis, and a list of the
%contributions to the state of the art.
%
%Parts of this chapter have been published in \cite{NegDeS:05-021}.

%\section{Background}\label{ch1_background}

City-scale events are getting more popular and attract a large number of people participating in various activities. For instance, on King’s Day, a national holiday in the Netherlands, a huge amount of people pour into the city and gather in the urban area, participating in various activities such as street parties, music festivals and boat parades. Event stakeholders, such as event organisers, police, municipalities, and crowd managers manage the crowd to avoid incidents. Crowd management practice consists of two phases \citep{martella2017current}, i.e. the planning phase and operational phase. In the planning phase, crowd managers require the past event data to infer guidelines, and to perform computer simulations of the crowds in the event.
